% Options for packages loaded elsewhere
\PassOptionsToPackage{unicode}{hyperref}
\PassOptionsToPackage{hyphens}{url}
\PassOptionsToPackage{dvipsnames,svgnames,x11names}{xcolor}
%
\documentclass[
  letterpaper,
  DIV=11,
  numbers=noendperiod]{scrartcl}

\usepackage{amsmath,amssymb}
\usepackage{iftex}
\ifPDFTeX
  \usepackage[T1]{fontenc}
  \usepackage[utf8]{inputenc}
  \usepackage{textcomp} % provide euro and other symbols
\else % if luatex or xetex
  \usepackage{unicode-math}
  \defaultfontfeatures{Scale=MatchLowercase}
  \defaultfontfeatures[\rmfamily]{Ligatures=TeX,Scale=1}
\fi
\usepackage{lmodern}
\ifPDFTeX\else  
    % xetex/luatex font selection
\fi
% Use upquote if available, for straight quotes in verbatim environments
\IfFileExists{upquote.sty}{\usepackage{upquote}}{}
\IfFileExists{microtype.sty}{% use microtype if available
  \usepackage[]{microtype}
  \UseMicrotypeSet[protrusion]{basicmath} % disable protrusion for tt fonts
}{}
\makeatletter
\@ifundefined{KOMAClassName}{% if non-KOMA class
  \IfFileExists{parskip.sty}{%
    \usepackage{parskip}
  }{% else
    \setlength{\parindent}{0pt}
    \setlength{\parskip}{6pt plus 2pt minus 1pt}}
}{% if KOMA class
  \KOMAoptions{parskip=half}}
\makeatother
\usepackage{xcolor}
\setlength{\emergencystretch}{3em} % prevent overfull lines
\setcounter{secnumdepth}{-\maxdimen} % remove section numbering
% Make \paragraph and \subparagraph free-standing
\ifx\paragraph\undefined\else
  \let\oldparagraph\paragraph
  \renewcommand{\paragraph}[1]{\oldparagraph{#1}\mbox{}}
\fi
\ifx\subparagraph\undefined\else
  \let\oldsubparagraph\subparagraph
  \renewcommand{\subparagraph}[1]{\oldsubparagraph{#1}\mbox{}}
\fi

\usepackage{color}
\usepackage{fancyvrb}
\newcommand{\VerbBar}{|}
\newcommand{\VERB}{\Verb[commandchars=\\\{\}]}
\DefineVerbatimEnvironment{Highlighting}{Verbatim}{commandchars=\\\{\}}
% Add ',fontsize=\small' for more characters per line
\usepackage{framed}
\definecolor{shadecolor}{RGB}{241,243,245}
\newenvironment{Shaded}{\begin{snugshade}}{\end{snugshade}}
\newcommand{\AlertTok}[1]{\textcolor[rgb]{0.68,0.00,0.00}{#1}}
\newcommand{\AnnotationTok}[1]{\textcolor[rgb]{0.37,0.37,0.37}{#1}}
\newcommand{\AttributeTok}[1]{\textcolor[rgb]{0.40,0.45,0.13}{#1}}
\newcommand{\BaseNTok}[1]{\textcolor[rgb]{0.68,0.00,0.00}{#1}}
\newcommand{\BuiltInTok}[1]{\textcolor[rgb]{0.00,0.23,0.31}{#1}}
\newcommand{\CharTok}[1]{\textcolor[rgb]{0.13,0.47,0.30}{#1}}
\newcommand{\CommentTok}[1]{\textcolor[rgb]{0.37,0.37,0.37}{#1}}
\newcommand{\CommentVarTok}[1]{\textcolor[rgb]{0.37,0.37,0.37}{\textit{#1}}}
\newcommand{\ConstantTok}[1]{\textcolor[rgb]{0.56,0.35,0.01}{#1}}
\newcommand{\ControlFlowTok}[1]{\textcolor[rgb]{0.00,0.23,0.31}{#1}}
\newcommand{\DataTypeTok}[1]{\textcolor[rgb]{0.68,0.00,0.00}{#1}}
\newcommand{\DecValTok}[1]{\textcolor[rgb]{0.68,0.00,0.00}{#1}}
\newcommand{\DocumentationTok}[1]{\textcolor[rgb]{0.37,0.37,0.37}{\textit{#1}}}
\newcommand{\ErrorTok}[1]{\textcolor[rgb]{0.68,0.00,0.00}{#1}}
\newcommand{\ExtensionTok}[1]{\textcolor[rgb]{0.00,0.23,0.31}{#1}}
\newcommand{\FloatTok}[1]{\textcolor[rgb]{0.68,0.00,0.00}{#1}}
\newcommand{\FunctionTok}[1]{\textcolor[rgb]{0.28,0.35,0.67}{#1}}
\newcommand{\ImportTok}[1]{\textcolor[rgb]{0.00,0.46,0.62}{#1}}
\newcommand{\InformationTok}[1]{\textcolor[rgb]{0.37,0.37,0.37}{#1}}
\newcommand{\KeywordTok}[1]{\textcolor[rgb]{0.00,0.23,0.31}{#1}}
\newcommand{\NormalTok}[1]{\textcolor[rgb]{0.00,0.23,0.31}{#1}}
\newcommand{\OperatorTok}[1]{\textcolor[rgb]{0.37,0.37,0.37}{#1}}
\newcommand{\OtherTok}[1]{\textcolor[rgb]{0.00,0.23,0.31}{#1}}
\newcommand{\PreprocessorTok}[1]{\textcolor[rgb]{0.68,0.00,0.00}{#1}}
\newcommand{\RegionMarkerTok}[1]{\textcolor[rgb]{0.00,0.23,0.31}{#1}}
\newcommand{\SpecialCharTok}[1]{\textcolor[rgb]{0.37,0.37,0.37}{#1}}
\newcommand{\SpecialStringTok}[1]{\textcolor[rgb]{0.13,0.47,0.30}{#1}}
\newcommand{\StringTok}[1]{\textcolor[rgb]{0.13,0.47,0.30}{#1}}
\newcommand{\VariableTok}[1]{\textcolor[rgb]{0.07,0.07,0.07}{#1}}
\newcommand{\VerbatimStringTok}[1]{\textcolor[rgb]{0.13,0.47,0.30}{#1}}
\newcommand{\WarningTok}[1]{\textcolor[rgb]{0.37,0.37,0.37}{\textit{#1}}}

\providecommand{\tightlist}{%
  \setlength{\itemsep}{0pt}\setlength{\parskip}{0pt}}\usepackage{longtable,booktabs,array}
\usepackage{calc} % for calculating minipage widths
% Correct order of tables after \paragraph or \subparagraph
\usepackage{etoolbox}
\makeatletter
\patchcmd\longtable{\par}{\if@noskipsec\mbox{}\fi\par}{}{}
\makeatother
% Allow footnotes in longtable head/foot
\IfFileExists{footnotehyper.sty}{\usepackage{footnotehyper}}{\usepackage{footnote}}
\makesavenoteenv{longtable}
\usepackage{graphicx}
\makeatletter
\def\maxwidth{\ifdim\Gin@nat@width>\linewidth\linewidth\else\Gin@nat@width\fi}
\def\maxheight{\ifdim\Gin@nat@height>\textheight\textheight\else\Gin@nat@height\fi}
\makeatother
% Scale images if necessary, so that they will not overflow the page
% margins by default, and it is still possible to overwrite the defaults
% using explicit options in \includegraphics[width, height, ...]{}
\setkeys{Gin}{width=\maxwidth,height=\maxheight,keepaspectratio}
% Set default figure placement to htbp
\makeatletter
\def\fps@figure{htbp}
\makeatother

\KOMAoption{captions}{tableheading}
\makeatletter
\@ifpackageloaded{caption}{}{\usepackage{caption}}
\AtBeginDocument{%
\ifdefined\contentsname
  \renewcommand*\contentsname{Table of contents}
\else
  \newcommand\contentsname{Table of contents}
\fi
\ifdefined\listfigurename
  \renewcommand*\listfigurename{List of Figures}
\else
  \newcommand\listfigurename{List of Figures}
\fi
\ifdefined\listtablename
  \renewcommand*\listtablename{List of Tables}
\else
  \newcommand\listtablename{List of Tables}
\fi
\ifdefined\figurename
  \renewcommand*\figurename{Figure}
\else
  \newcommand\figurename{Figure}
\fi
\ifdefined\tablename
  \renewcommand*\tablename{Table}
\else
  \newcommand\tablename{Table}
\fi
}
\@ifpackageloaded{float}{}{\usepackage{float}}
\floatstyle{ruled}
\@ifundefined{c@chapter}{\newfloat{codelisting}{h}{lop}}{\newfloat{codelisting}{h}{lop}[chapter]}
\floatname{codelisting}{Listing}
\newcommand*\listoflistings{\listof{codelisting}{List of Listings}}
\makeatother
\makeatletter
\makeatother
\makeatletter
\@ifpackageloaded{caption}{}{\usepackage{caption}}
\@ifpackageloaded{subcaption}{}{\usepackage{subcaption}}
\makeatother
\ifLuaTeX
  \usepackage{selnolig}  % disable illegal ligatures
\fi
\usepackage{bookmark}

\IfFileExists{xurl.sty}{\usepackage{xurl}}{} % add URL line breaks if available
\urlstyle{same} % disable monospaced font for URLs
\hypersetup{
  pdftitle={Homework 3 PDF},
  pdfauthor={Katerina Bischel},
  colorlinks=true,
  linkcolor={blue},
  filecolor={Maroon},
  citecolor={Blue},
  urlcolor={Blue},
  pdfcreator={LaTeX via pandoc}}

\title{Homework 3 PDF}
\author{Katerina Bischel}
\date{}

\begin{document}
\maketitle

\subsection{}\label{section}

\begin{Shaded}
\begin{Highlighting}[]
\CommentTok{\#https://github.com/katerinabischel/bischel{-}katerina\_homework{-}03.git}
\CommentTok{\# general use}
\FunctionTok{library}\NormalTok{(tidyverse)}
\end{Highlighting}
\end{Shaded}

\begin{verbatim}
-- Attaching core tidyverse packages ------------------------ tidyverse 2.0.0 --
v dplyr     1.1.4     v readr     2.1.5
v forcats   1.0.0     v stringr   1.5.1
v ggplot2   3.5.1     v tibble    3.2.1
v lubridate 1.9.3     v tidyr     1.3.1
v purrr     1.0.2     
-- Conflicts ------------------------------------------ tidyverse_conflicts() --
x dplyr::filter() masks stats::filter()
x dplyr::lag()    masks stats::lag()
i Use the conflicted package (<http://conflicted.r-lib.org/>) to force all conflicts to become errors
\end{verbatim}

\begin{Shaded}
\begin{Highlighting}[]
\FunctionTok{library}\NormalTok{(readxl)}
\FunctionTok{library}\NormalTok{(here)}
\end{Highlighting}
\end{Shaded}

\begin{verbatim}
here() starts at /Users/ktdroppa/Desktop/ENVS-193DS/bischel-katerina_homework-03
\end{verbatim}

\begin{Shaded}
\begin{Highlighting}[]
\FunctionTok{library}\NormalTok{(janitor)}
\end{Highlighting}
\end{Shaded}

\begin{verbatim}

Attaching package: 'janitor'

The following objects are masked from 'package:stats':

    chisq.test, fisher.test
\end{verbatim}

\begin{Shaded}
\begin{Highlighting}[]
\CommentTok{\# visualizing pairs}
\FunctionTok{library}\NormalTok{(GGally)}
\end{Highlighting}
\end{Shaded}

\begin{verbatim}
Registered S3 method overwritten by 'GGally':
  method from   
  +.gg   ggplot2
\end{verbatim}

\begin{Shaded}
\begin{Highlighting}[]
\CommentTok{\# model selection}
\FunctionTok{library}\NormalTok{(MuMIn)}

\CommentTok{\# model predictions}
\FunctionTok{library}\NormalTok{(ggeffects)}

\CommentTok{\# model tables}
\FunctionTok{library}\NormalTok{(gtsummary)}
\FunctionTok{library}\NormalTok{(flextable)}
\end{Highlighting}
\end{Shaded}

\begin{verbatim}

Attaching package: 'flextable'

The following objects are masked from 'package:gtsummary':

    as_flextable, continuous_summary

The following object is masked from 'package:purrr':

    compose
\end{verbatim}

\begin{Shaded}
\begin{Highlighting}[]
\FunctionTok{library}\NormalTok{(modelsummary)}
\end{Highlighting}
\end{Shaded}

\begin{verbatim}
`modelsummary` 2.0.0 now uses `tinytable` as its default table-drawing
  backend. Learn more at: https://vincentarelbundock.github.io/tinytable/

Revert to `kableExtra` for one session:

  options(modelsummary_factory_default = 'kableExtra')
  options(modelsummary_factory_latex = 'kableExtra')
  options(modelsummary_factory_html = 'kableExtra')

Silence this message forever:

  config_modelsummary(startup_message = FALSE)
\end{verbatim}

\begin{Shaded}
\begin{Highlighting}[]
\NormalTok{drought\_exp }\OtherTok{\textless{}{-}} \FunctionTok{read\_xlsx}\NormalTok{(}\AttributeTok{path =} \FunctionTok{here}\NormalTok{(}\StringTok{"data"}\NormalTok{, }
                                     \StringTok{"Valliere\_etal\_EcoApps\_Data.xlsx"}\NormalTok{),}
                         \AttributeTok{sheet =} \StringTok{"First Harvest"}\NormalTok{)}

\CommentTok{\# quick look at data }
\FunctionTok{str}\NormalTok{(drought\_exp)}
\end{Highlighting}
\end{Shaded}

\begin{verbatim}
tibble [70 x 13] (S3: tbl_df/tbl/data.frame)
 $ Species            : chr [1:70] "ENCCAL" "ENCCAL" "ENCCAL" "ENCCAL" ...
 $ Water              : chr [1:70] "WW" "WW" "WW" "WW" ...
 $ Rep #              : num [1:70] 1 2 3 4 5 1 2 3 4 5 ...
 $ Height (cm)        : num [1:70] 5.8 4.9 8.4 6.5 7.1 3.2 4.4 4.2 4.5 3.9 ...
 $ Leaf #             : num [1:70] 11 8 11 12 10 7 7 10 8 6 ...
 $ Leaf dry weight (g): num [1:70] 0.0294 0.0185 0.0177 0.0178 0.0164 0.017 0.0193 0.0153 0.0159 0.0133 ...
 $ Leaf area (cm2)    : num [1:70] 5.01 3.98 3.69 3.84 3.63 3.06 3.1 2.94 2.73 2.61 ...
 $ SLA                : num [1:70] 170 215 209 216 222 ...
 $ Total LA           : num [1:70] 55.1 31.8 40.6 46.1 36.3 ...
 $ Shoot (g)          : num [1:70] 0.253 0.164 0.241 0.213 0.232 ...
 $ Root (g)           : num [1:70] 0.202 0.165 0.209 0.146 0.12 ...
 $ Total (g)          : num [1:70] 0.455 0.329 0.45 0.359 0.352 ...
 $ R:S                : num [1:70] 0.8 1 0.9 0.7 0.5 0.8 1.2 3.1 0.9 1.2 ...
\end{verbatim}

\begin{Shaded}
\begin{Highlighting}[]
\FunctionTok{class}\NormalTok{(drought\_exp)}
\end{Highlighting}
\end{Shaded}

\begin{verbatim}
[1] "tbl_df"     "tbl"        "data.frame"
\end{verbatim}

\begin{Shaded}
\begin{Highlighting}[]
\NormalTok{drought\_exp\_clean }\OtherTok{\textless{}{-}}\NormalTok{ drought\_exp }\SpecialCharTok{|\textgreater{}} 
  \FunctionTok{clean\_names}\NormalTok{() }\SpecialCharTok{|\textgreater{}}   \CommentTok{\# Standardize column names}
  \FunctionTok{mutate}\NormalTok{(}
    \CommentTok{\# Decode categorical variables with more descriptive names}
    \AttributeTok{species =} \FunctionTok{as.factor}\NormalTok{(species),}
    \AttributeTok{water =} \FunctionTok{as.factor}\NormalTok{(water),}
    \AttributeTok{species\_name =} \FunctionTok{case\_when}\NormalTok{(}
\NormalTok{      species }\SpecialCharTok{==} \StringTok{"ENCCAL"} \SpecialCharTok{\textasciitilde{}} \StringTok{"Encelia californica"}\NormalTok{,  }\CommentTok{\# Bush sunflower}
\NormalTok{      species }\SpecialCharTok{==} \StringTok{"ESCCAL"} \SpecialCharTok{\textasciitilde{}} \StringTok{"Eschscholzia californica"}\NormalTok{,  }\CommentTok{\# California poppy}
\NormalTok{      species }\SpecialCharTok{==} \StringTok{"PENCEN"} \SpecialCharTok{\textasciitilde{}} \StringTok{"Penstemon centranthifolius"}\NormalTok{,  }\CommentTok{\# Scarlet bugler}
\NormalTok{      species }\SpecialCharTok{==} \StringTok{"GRICAM"} \SpecialCharTok{\textasciitilde{}} \StringTok{"Grindelia camporum"}\NormalTok{,  }\CommentTok{\# Great valley gumweed}
\NormalTok{      species }\SpecialCharTok{==} \StringTok{"SALLEU"} \SpecialCharTok{\textasciitilde{}} \StringTok{"Salvia leucophylla"}\NormalTok{,  }\CommentTok{\# Purple sage}
\NormalTok{      species }\SpecialCharTok{==} \StringTok{"STIPUL"} \SpecialCharTok{\textasciitilde{}} \StringTok{"Nasella pulchra"}\NormalTok{,  }\CommentTok{\# Purple needlegrass}
\NormalTok{      species }\SpecialCharTok{==} \StringTok{"LOTSCO"} \SpecialCharTok{\textasciitilde{}} \StringTok{"Acmispon glaber"}  \CommentTok{\# Deerweed}
\NormalTok{    ),}
    \AttributeTok{water\_treatment =} \FunctionTok{case\_when}\NormalTok{(}
\NormalTok{      water }\SpecialCharTok{==} \StringTok{"WW"} \SpecialCharTok{\textasciitilde{}} \StringTok{"Well watered"}\NormalTok{,}
\NormalTok{      water }\SpecialCharTok{==} \StringTok{"DS"} \SpecialCharTok{\textasciitilde{}} \StringTok{"Drought stressed"}
\NormalTok{    )}
\NormalTok{  ) }\SpecialCharTok{|\textgreater{}} 
  \FunctionTok{relocate}\NormalTok{(species\_name, }\AttributeTok{.after =}\NormalTok{ species) }\SpecialCharTok{|\textgreater{}}   \CommentTok{\# Move species\_name after species column}
  \FunctionTok{relocate}\NormalTok{(water\_treatment, }\AttributeTok{.after =}\NormalTok{ water) }\SpecialCharTok{|\textgreater{}}   \CommentTok{\# Move water\_treatment after water column}
  \CommentTok{\# Optional: Check for and handle missing values if necessary}
  \CommentTok{\# mutate(across(where(is.numeric), \textasciitilde{}if\_else(is.na(.), median(., na.rm = TRUE), .)))}
  \CommentTok{\# For an overview of the cleaned data}
  \FunctionTok{glimpse}\NormalTok{()}
\end{Highlighting}
\end{Shaded}

\begin{verbatim}
Rows: 70
Columns: 15
$ species           <fct> ENCCAL, ENCCAL, ENCCAL, ENCCAL, ENCCAL, ENCCAL, ENCC~
$ species_name      <chr> "Encelia californica", "Encelia californica", "Encel~
$ water             <fct> WW, WW, WW, WW, WW, DS, DS, DS, DS, DS, WW, WW, WW, ~
$ water_treatment   <chr> "Well watered", "Well watered", "Well watered", "Wel~
$ rep_number        <dbl> 1, 2, 3, 4, 5, 1, 2, 3, 4, 5, 1, 2, 3, 4, 5, 1, 2, 3~
$ height_cm         <dbl> 5.8, 4.9, 8.4, 6.5, 7.1, 3.2, 4.4, 4.2, 4.5, 3.9, 13~
$ leaf_number       <dbl> 11, 8, 11, 12, 10, 7, 7, 10, 8, 6, 17, 13, 26, 23, 1~
$ leaf_dry_weight_g <dbl> 0.0294, 0.0185, 0.0177, 0.0178, 0.0164, 0.0170, 0.01~
$ leaf_area_cm2     <dbl> 5.01, 3.98, 3.69, 3.84, 3.63, 3.06, 3.10, 2.94, 2.73~
$ sla               <dbl> 170.41, 214.95, 208.66, 215.92, 221.54, 180.20, 160.~
$ total_la          <dbl> 55.11, 31.81, 40.63, 46.12, 36.33, 21.44, 21.70, 29.~
$ shoot_g           <dbl> 0.2527, 0.1636, 0.2411, 0.2134, 0.2319, 0.1804, 0.12~
$ root_g            <dbl> 0.2021, 0.1649, 0.2092, 0.1456, 0.1203, 0.1515, 0.15~
$ total_g           <dbl> 0.4548, 0.3285, 0.4503, 0.3590, 0.3522, 0.3319, 0.27~
$ r_s               <dbl> 0.8, 1.0, 0.9, 0.7, 0.5, 0.8, 1.2, 3.1, 0.9, 1.2, 0.~
\end{verbatim}

\begin{Shaded}
\begin{Highlighting}[]
\CommentTok{\# Optionally, check for missing values}
\FunctionTok{sum}\NormalTok{(}\FunctionTok{is.na}\NormalTok{(drought\_exp\_clean))}
\end{Highlighting}
\end{Shaded}

\begin{verbatim}
[1] 0
\end{verbatim}



\end{document}
